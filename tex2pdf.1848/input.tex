\documentclass[]{article}
\usepackage{lmodern}
\usepackage{amssymb,amsmath}
\usepackage{ifxetex,ifluatex}
\usepackage{fixltx2e} % provides \textsubscript
\ifnum 0\ifxetex 1\fi\ifluatex 1\fi=0 % if pdftex
  \usepackage[T1]{fontenc}
  \usepackage[utf8]{inputenc}
\else % if luatex or xelatex
  \ifxetex
    \usepackage{mathspec}
    \usepackage{xltxtra,xunicode}
  \else
    \usepackage{fontspec}
  \fi
  \defaultfontfeatures{Mapping=tex-text,Scale=MatchLowercase}
  \newcommand{\euro}{€}
\fi
% use upquote if available, for straight quotes in verbatim environments
\IfFileExists{upquote.sty}{\usepackage{upquote}}{}
% use microtype if available
\IfFileExists{microtype.sty}{%
\usepackage{microtype}
\UseMicrotypeSet[protrusion]{basicmath} % disable protrusion for tt fonts
}{}
\usepackage[margin=1in]{geometry}
\usepackage{longtable,booktabs}
\usepackage{graphicx}
\makeatletter
\def\maxwidth{\ifdim\Gin@nat@width>\linewidth\linewidth\else\Gin@nat@width\fi}
\def\maxheight{\ifdim\Gin@nat@height>\textheight\textheight\else\Gin@nat@height\fi}
\makeatother
% Scale images if necessary, so that they will not overflow the page
% margins by default, and it is still possible to overwrite the defaults
% using explicit options in \includegraphics[width, height, ...]{}
\setkeys{Gin}{width=\maxwidth,height=\maxheight,keepaspectratio}
\ifxetex
  \usepackage[setpagesize=false, % page size defined by xetex
              unicode=false, % unicode breaks when used with xetex
              xetex]{hyperref}
\else
  \usepackage[unicode=true]{hyperref}
\fi
\hypersetup{breaklinks=true,
            bookmarks=true,
            pdfauthor={Anja Leskovšek},
            pdftitle={Porocilo pri predmetu Analiza podatkov s programom R},
            colorlinks=true,
            citecolor=blue,
            urlcolor=blue,
            linkcolor=magenta,
            pdfborder={0 0 0}}
\urlstyle{same}  % don't use monospace font for urls
\setlength{\parindent}{0pt}
\setlength{\parskip}{6pt plus 2pt minus 1pt}
\setlength{\emergencystretch}{3em}  % prevent overfull lines
\setcounter{secnumdepth}{0}

%%% Use protect on footnotes to avoid problems with footnotes in titles
\let\rmarkdownfootnote\footnote%
\def\footnote{\protect\rmarkdownfootnote}

%%% Change title format to be more compact
\usepackage{titling}

% Create subtitle command for use in maketitle
\newcommand{\subtitle}[1]{
  \posttitle{
    \begin{center}\large#1\end{center}
    }
}

\setlength{\droptitle}{-2em}
  \title{Porocilo pri predmetu Analiza podatkov s programom R}
  \pretitle{\vspace{\droptitle}\centering\huge}
  \posttitle{\par}
  \author{Anja Leskovšek}
  \preauthor{\centering\large\emph}
  \postauthor{\par}
  \date{}
  \predate{}\postdate{}

\usepackage[slovene]{babel}
\usepackage{graphicx}


\begin{document}

\maketitle


\section{Izbira teme}\label{izbira-teme}

Moja tema je analiza podatkov hokejske lige NHL za leto 15/14. V mojem
projektu bom naredilarazlicne primerjave in analize. Najbolj me zanima
analiza njihove statistike glede na njihov rojstni datum. Glede na
raziskave (\url{http://www.bbc.com/sport/olympics/18891749}) naj bi bili
hokejisti rojeni januarja, febuarja ali marca boljši. Primerjala bom
torej njihovo statistiko in mesec rojstva, ter ugotovila ali raziskava
tudi v mojem primeru drži.

\section{Obdelava, uvoz in cišcenje
podatkov}\label{obdelava-uvoz-in-ciscenje-podatkov}

POVEZAVE:

\begin{itemize}
\itemsep1pt\parskip0pt\parsep0pt
\item
  \url{http://www.nhl.com/stats/player?reportType=season\&report=skatersummary\&season=20142015\&gameType=2\&sort=points\&aggregate=1\&pos=S}
\item
  \url{http://www.nhl.com/stats/player?reportType=season\&report=bios\&season=20142015\&gameType=2\&sort=playerBirthDate\&aggregate=1\&pos=S}
\end{itemize}

Podatke, ki sem jih uvozila so bili v obliki JSON.

Na prvem linku imamo veliko statistik za 882 igralcev, vendar za moj
projekt niso vse pomembne, zato sem nekaj stolpcev tudi odstranila.
Ostala mi je tabela z naslednjimi stolpci:

\begin{itemize}
\itemsep1pt\parskip0pt\parsep0pt
\item
  imena igralcev (imenska spremenljivka)
\item
  odigrane tekme ( številska spremenljivka): koliko tekem je posamezni
  igralec odigral v sezoni
\item
  streli ( številska spremenljivka ): koliko strelov je posamezni
  igralec imel v sezoni
\item
  goli ( številska spremenljivka ): koliko golov je posamezni igralec
  dal v sezoni
\item
  asistence ( številska spremenljivka ): koliko asistenc je posamezni
  igralec imel v sezoni
\item
  tocke (številska spremenljivka): koliko tock je dal posamezni igralec
  v sezoni
\item
  igralni položaj ( imenska spremenljivka ): na katerem položaju je
  igral igralec -\textgreater{} D=branilec, L=levo krilo, R=desno krilo,
  C=center.
\item
  strelni procent (številska spremenljivka): koliko golov je posamezni
  igralec zadel glede na strele v sezoni
\end{itemize}

Te stolpce sem preuredila in preimenovala v slovenšcino. Za procent
strela sem sama napisala funkcijo za izracun in dodala h tabeli ter
odstranila tiste vrstice, ki so vsebovale NA. Ker so nekateri igralci
zelo malo igrali sem odstranila vse tiste vrstice, kjer so bile odigrane
tekme pod 15. Potem sem tabelo razvrstila po abecednem vrstnem redu
imen. Poleg prvotne tabele sem naredila še dodatno tabelo, ki sem jo
imenovala povprecja, kjer sem izracunala povprecje vsake podkategorije.

Na drugem linku pa je bilo veliko razlicnih podatkov za iste igralce,
vendar ko sem jih uvozila sem odstranila vse stolpce in so mi ostali le:
- igralci - višina igralcev - rojstni datum

Rojstni datum sem spremenila v dva stolpca, ki vsebujeta le letnico in
mesec rojstva. To tabelo nato dodala k prvotni tabeli,

Naredila sem nekaj grafov z plotly in ggplot2.

\includegraphics{projekt_files/figure-latex/slika6-1.pdf}

Slika prikazuje koliko tock je dal posamezni igralec. Ravna crta pa
prikazuje povprecje vseh tock v sezoni.

\includegraphics{projekt_files/figure-latex/slika7-1.pdf}

Slika prikazuje koliko golov je dal posamezni igralec. Ravna crta pa
prikazuje povprecje vseh golov v sezoni.

\includegraphics{projekt_files/figure-latex/slika8-1.pdf}

Slika prikazuje koliko asistenc je dal posamezni igralec. Ravna crta pa
prikazuje povprecje vseh asistenc v sezoni.

\includegraphics{projekt_files/figure-latex/slika9-1.pdf}

Slika prikazuje koliko strelov je dal posamezni igralec. Ravna crta pa
prikazuje povprecje vseh strelov v sezoni.

\includegraphics{projekt_files/figure-latex/slika10-1.pdf}

Slika prikazuje višino posameznega igralca. Ravna crta pa prikazuje
povprecje.

\section{Analiza in vizualizacija
podatkov}\label{analiza-in-vizualizacija-podatkov}

\includegraphics{projekt_files/figure-latex/vizualizacija-1.pdf}
\includegraphics{projekt_files/figure-latex/vizualizacija-2.pdf}

Za analizo podatkov sem najprej pogledala, ce so tisti igralci boljši,
ki so rojeni v prvem polletju. Naredila sem dve tabeli, za igralce v
prvem polletju in za igralce v drugem polletju. Za vsako kategorijo sem
izracunala povprecje.

\begin{longtable}[c]{@{}rrrrrr@{}}
\toprule
odigrane.tekme & streli & goli & asistence & tocke &
procent.strela\tabularnewline
\midrule
\endhead
61.85379 & 100.2454 & 8.75718 & 14.94778 & 23.70496 &
7.797467\tabularnewline
\bottomrule
\end{longtable}

\begin{longtable}[c]{@{}rrrrrr@{}}
\toprule
odigrane.tekme & streli & goli & asistence & tocke &
procent.strela\tabularnewline
\midrule
\endhead
64.03618 & 112.5066 & 10.375 & 17.80263 & 28.17763 &
8.192796\tabularnewline
\bottomrule
\end{longtable}

Na sliki sem pokazala koliko tock so dali igralci iz prvega in drugega
polletja ter njihovo povprecje. Igralce iz prvega polletja sem oznacila
z rdeco barvo, igralce iz drugega polletja pa z modro.

\includegraphics{projekt_files/figure-latex/slika11-1.pdf}

Opazimo, da je vec modrih pikic v zgornjem delu grafa, kar pomeni, da je
vec igralcev iz drugega polletja dalo vec tock.

Za boljšo analizo sem podatke razvrstila po doseženih tockah in izbrala
100 igralcev z najvec doseženimi tockami. Ponovno sem jih razvrstila v
dve polletji in izracunala skupne dosežene tocke. Na spodnjem grafu
vidimo, da so vec tock dosegli igralci rojeni v drugem polletju.

\includegraphics{projekt_files/figure-latex/slika12-1.pdf}

Ocitno je, da so vec tock dosegli igralci rojeni v drugem polletju.

Prišla sem do zakljucka, da mesec rojstva ne vpliva na boljši rezultat
igralcev pri profesionalni ligi kot pa pri mlajših igralcih.

Pri naslednji analizi sem naredila novo tabelo, kjer sem za stolpce dala
igralce, državo, koliko tekem so odigral in koliko tock so dal. Potem
sem dodala nov stolpec, kjer sem izracunala koliko tock je igralec v
povprecju dal na tekmo. Iz tabele sem nato izbrala le prvih 8 igralcev,
ki so imeli najvišjo število tock glede na odigrano tekmo in tako sem
dobila osem najboljših igralcev sezone. Spodaj je prikazana tabela.

\begin{longtable}[c]{@{}llrrr@{}}
\toprule
igralci & drzava & odigrane.tekme & tocke &
stevilo.tock.na.tekmo\tabularnewline
\midrule
\endhead
Sidney Crosby & CAN & 77 & 84 & 1.090909\tabularnewline
Tyler Seguin & CAN & 71 & 77 & 1.084507\tabularnewline
Jamie Benn & CAN & 82 & 87 & 1.060976\tabularnewline
Patrick Kane & USA & 61 & 64 & 1.049180\tabularnewline
John Tavares & CAN & 82 & 86 & 1.048780\tabularnewline
Pavel Datsyuk & RUS & 63 & 65 & 1.031746\tabularnewline
\bottomrule
\end{longtable}

Naslednji zemljevid prikazuje koliko igralcev je iz razlicnih držav.

\includegraphics{projekt_files/figure-latex/zem1-1.pdf}

Opazimo, da je najvec igralcev iz Kanade in iz Združenih držav Amerike.

V naslednji tabeli pa sem prikazala koliko tock skupaj so dali igralci
iz vsake države. Naredila sem tudi zemljevid za boljšo predstavo.

\begin{longtable}[c]{@{}llrrrrrrrrlr@{}}
\toprule
igralci & drzava & letnica.rojstva & mesec.rojstva & visina & streli &
goli & asistence & tocke & procent.strela & igralni.polozaj &
odigrane.tekme\tabularnewline
\midrule
\endhead
Aaron Ekblad & CAN & 1996 & 2 & 76 & 170 & 12 & 27 & 39 & 7.06 & D &
81\tabularnewline
Adam Burish & USA & 1983 & 1 & 73 & 22 & 1 & 2 & 3 & 4.55 & R &
20\tabularnewline
Adam Clendening & USA & 1992 & 10 & 72 & 17 & 1 & 3 & 4 & 5.88 & D &
21\tabularnewline
Adam Cracknell & CAN & 1985 & 7 & 74 & 17 & 0 & 1 & 1 & 0.00 & R &
17\tabularnewline
Adam Henrique & CAN & 1990 & 2 & 72 & 127 & 16 & 27 & 43 & 12.60 & C &
75\tabularnewline
Adam Larsson & SWE & 1992 & 11 & 75 & 91 & 3 & 21 & 24 & 3.30 & D &
64\tabularnewline
\bottomrule
\end{longtable}

\includegraphics{projekt_files/figure-latex/zem2-1.pdf}

\end{document}
